\section{MARCO TEORICO} 
\begin {itemize}
\subsection{Entity Framework}
	\item Entity Framework (EF) es un mapeador relacional de objetos (O/RM) probado y probado para .NET con muchos años de desarrollo y estabilización de características.\\
Es la tecnología de acceso a datos recomendada por Microsoft para nuevas aplicaciones.\\	
	\item Como O/RM, EF reduce la discrepancia de impedancia entre los mundos relacionales y orientados a objetos, permitiendo a los desarrolladores escribir aplicaciones que interactúan con datos almacenados en bases de datos relacionales utilizando objetos .NET de tipo fuerte que representan el dominio de la aplicación y eliminando la necesidad para una gran parte del código de "plumbing" de acceso a datos que normalmente necesitan escribir.\\
	\item EF implementa muchas características populares de O/RM:\\
	- Mapeo de clases de entidad POCO que no dependen de ningún tipo de EF\\
	- Seguimiento automático de cambios\\
	- Resolución de identidad y Unidad de Trabajo.\\
	- Carga ansiosa, perezosa y explícita.\\
	- Traducción de consultas fuertemente tipadas utilizando LINQ \\
	- Capacidades de mapeo enriquecidas, incluyendo soporte para:\\
	\subitem - Relaciones uno a uno, uno a muchos y muchos a muchos\\
	\subitem - Herencia (tabla por jerarquía, tabla por tipo y tabla por clase concreta)\\
	\subitem - Tipos complejos\\
	\subitem - Procedimientos almacenados\\
	\newline
	- Un diseñador visual para crear modelos de entidad.\\
	- Una experiencia de "Código Primero" para crear modelos de entidad al escribir código.\\
	- Los modelos pueden generarse a partir de bases de datos existentes y luego editarse manualmente, o pueden crearse desde cero y luego usarse para generar nuevas bases de datos.\\
	- Integración con modelos de aplicaciones de .NET Framework, incluido ASP.NET, y mediante enlace de datos, con WPF y WinForms.\\
	- Conectividad de base de datos basada en ADO.NET y numerosos proveedores disponibles para conectarse a SQL Server, Oracle, MySQL, SQLite, PostgreSQL, DB2, etc.\\
	\newpage
\subsection {Pruebas Unitarias}
\item Una prueba unitaria se utiliza para comprobar que un método concreto del código de producción funciona correctamente, probar las regresiones o realizar pruebas relacionadas (buddy) o de humo. Una prueba por orden se utiliza para ejecutar otras pruebas en un orden especificado. 
\item Características:
\item Para que una prueba unitaria tenga la calidad suficiente se deben cumplir los siguientes requisitos:
\item Automatizable\\
 No debería requerirse una intervención manual. Esto es especialmente útil para integración continua.
\item Completas\\
 Deben cubrir la mayor cantidad de código.
\item Repetibles o Reutilizables\\
No se deben crear pruebas que sólo puedan ser ejecutadas una sola vez. También es útil para integración continua.
\item Independientes\\
 La ejecución de una prueba no debe afectar a la ejecución de otra.
\item Profesionales\\
 Las pruebas deben ser consideradas igual que el código, con la misma profesionalidad, documentación, etc.
Aunque estos requisitos no tienen que ser cumplidos al pie de la letra, se recomienda seguirlos o de lo contrario las pruebas pierden parte de su función.
\end{itemize}
